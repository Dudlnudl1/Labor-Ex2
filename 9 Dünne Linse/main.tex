\documentclass[12pt,a4paper,twoside]{article}
\usepackage{labor}
\begin{document}

%fill for cover and header creation
\newcommand\laboratorynumber{2}
\title{Dünne Linsen}
\newcommand\supervisor{Ditlbacher, Harald}
\newcommand\groupnumber{42}

\newcommand\participantonelastname{Eisner}
\newcommand\participantonefirstname{Nico}
\newcommand\participantoneid{12214121}
\newcommand\participanttwolastname{Waldl}
\newcommand\participanttwofirstname{Philip}
\newcommand\participanttwoid{12214120}
\author{\participantonelastname \ \& \participanttwolastname}

\newcommand\degreeid{UB 033 678}
\newcommand\semester{23WS}
\date{17.11.2023}

%select correct course title
%\newcommand\coursetitle{Einführung in die \\ physikalischen Messmethoden}
%\newcommand\coursetitle{Laborübungen 1: \\ Mechanik und Wärme}
\newcommand\coursetitle{Laborübungen 2: \\ Elektrizität, Magnetismus, Optik}
%\newcommand\coursetitle{Fortgeschrittenen Praktikum 1: \\ Technische Physik}
%\newcommand\coursetitle{Fortgeschrittenen Praktikum 2: \\ Allgemeine Physik}

%\input{cover}
%\maketitle %short title alternative

\includepdf[pages={1}]{../Deckblätter/Deckblatt_Dünne_Linsen.pdf}

\tableofcontents
\newpage

\section{Aufgabenstellung} %jo beschreibn wos gmocht host ------------------------------
Im Experiment Dünne Linsen gilt es Brennweite einer Sammellinse über zwei verschiedene Metoden zu bestimmen. 
Die Erste Methode ist jene nach der Laplace'sche Methode, die zweite nach dem Bessel'schen Verfahren. 
Desweiteren ist die Brennweite einer Zerstreuungslinse zu bestimmen. 
Zum schluss sollen noch einige Linsenfehler veranschaulicht werden. 
\\
\\
Gesuche Größen: 
\begin{itemize}
    \item Brennweite Laplace Sammellinse
    \item Brennweite Bessel Sammellinse
    \item Brennweite Zerstreuungslinse
\end{itemize}

\noindent
Alle Informationen und Methodiken wurden uns von der Technischen Universität bereitgestellt \cite{teachcenter2}. 

\section{Voraussetzungen \& Grundlagen} %Grundlagen erklären, Formeln mit erklärung
Für die Bestimmung der Brennweite einer Sammellinse gibt es, wie im vorherigen Kapitel erwähnt, mehrere Verfahren. 
\\
\\
Bei der Laplace'sche Methode Methode wird durch Messung der Längen $g$ und $b$ bei fokussierten Bild die Brennweite $f$ bestimmt. 
In der folgenden Abbildung \ref{fig:laplace_theorie} sieht man die Längen dargestellt. Dabei ist die Bildweite $b$ jene zwischen Linse und Schirm und die Gegenstandweite $g$ jene zwischen Projektor und Linse. 

\begin{figure}[H]
    \centering
    \includegraphics[width=0.6\linewidth]{nudes/laplace_theorie.jpg}
    \caption{Theoretischer Aufbau der Laplace'schen Methode. Bild aus Skriptum Dünne Linsen Seite 2 entnommen \cite{teachcenter2}. }
    \label{fig:laplace_theorie}
\end{figure}

\noindent
Mit der Formel \ref{eq:laplace} lässt sich aus der Bildweite $b$ und der Gegenstandweite $g$ die Brennweite $f$ berrechnen. 

\begin{equation}
    \label{eq:laplace}
    \centerline{$\frac{1}{f}=\frac{1}{g} + \frac{1}{b}$}
\end{equation}

\noindent
Ähnlich sieht es bei dem Bessel'schen Verfahren aus. Das Bild wird fokussiert. Durch bestimmung der Bildweite $b$ und der Gegenstandweite $g$ lässt sich der Gesamtabstand $a = g+b$ berrechnen. 
Die Linse wird verschoben, bis das Bild erneut scharf zu erkennen ist. Diese Strecke ist die Verschiebung $e$ wie in Abbildung \ref{fig:bessel_theorie} zu erkennen ist.  

\begin{figure}[H]
    \centering
    \includegraphics[width=0.6\linewidth]{nudes/bessel_theorie.jpg}
    \caption{Theoretischer Aufbau des Bessel'schen Verfahrens. Bild aus Skriptum Dünne Linsen Seite 3 entnommen \cite{teachcenter2}. }
    \label{fig:bessel_theorie}
\end{figure}

\noindent
Durch die Formel \ref{eq:bessel} lässt sich die Brennweite $f$ berrechnen. 

\begin{equation}
    \label{eq:bessel}
    \centerline{$\frac{1}{f}=\frac{1}{4} (\frac{a^2 - e^2}{a})$}
\end{equation}

\noindent
Um die Brennweite $f_s$ einer Zerstreuungslinse zu bestimmen benötigt man die Gegenstandweite $g'$ sowie die Bildweite $b$. 
Um diese zu bestimmen, wird die Linse in einer Gegenstandweite $g'$ zum Schirm aufgestellt. Durch Verschieben des Schirmes, bis das Bild scharf zu erkennen ist wird die Bildweite $b$ bestimmt. Die Brennweite $f_s$ wird mit folgender Formel bestimmt. 

\begin{equation}
    \label{eq:Zerstreuungslinse}
    \centerline{$\frac{1}{f_s}=\frac{1}{g'} + \frac{1}{b}$}
\end{equation}
\noindent
Hierbei ist jedoch anzumerken, dass die Brennweite $f_s$ ein negativer Wert ist. 

\section{Versuchsanordnung} %mit skizze kurz beschreiben ------------------------------

    \begin{figure}[H]
        \centering
        %\includegraphics[width=0.6\linewidth, angle=-90]{nudes/bild.jpg}
        \caption{müll}
        \label{fig:müllbild}
    \end{figure}

\section{Geräteliste} %jo holt a listn ------------------------------

    \begin{table}[H]
        \centering
        \caption{Im Versuch verwendete Geräte und Utensilien.}
        \label{tab:geraete}
        \begin{tabular}{| l | l | l | l |}
            \hline
            Gerät   & Typ   & Gerätenummer  & Unsicherheit \\
            \hline
        \end{tabular}
    \end{table}


\section{Versuchsdurchführung \& Messergebnisse} %nachvollziehbar und klar dargestellt ------------------------------


\section{Auswertung und Unsicherheitsanalyse} %Nicht nur zahlen angeben ------------------------------

In der Auswertung werden zur erhöhten Genauigkeit durchgehend ungerundete Werte bis zu den Endergebnissen verwendet und nur zur Darstellung gerundet. \\
Zur Berechnung der Unsicherheiten wird, wenn nicht anders angegeben, die Größtunsicherheitsmethode verwendet.


\section{Diskussion} %diskussion der Unsicherheiten und Ergebnisse und evtl. verlgeich mit Literatur ------------------------------


\section{Zusammenfassung} %klare, übersichtliche vollständige beantwortung der Aufgabenstellung ------------------------------


\printbibliography[heading=bibintoc]
\end{document}
