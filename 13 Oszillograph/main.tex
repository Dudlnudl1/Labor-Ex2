\documentclass[12pt,a4paper,twoside]{article}
\usepackage{labor}
\begin{document}

%fill for cover and header creation
\newcommand\laboratorynumber{2}
\title{Oszillograph}
\newcommand\supervisor{Ditlbacher, Harald}
\newcommand\groupnumber{42}

\newcommand\participantonelastname{Eisner}
\newcommand\participantonefirstname{Nico}
\newcommand\participantoneid{12214121}
\newcommand\participanttwolastname{Waldl}
\newcommand\participanttwofirstname{Philip}
\newcommand\participanttwoid{12214120}
\author{\participantonelastname \ \& \participanttwolastname}

\newcommand\degreeid{UB 033 678}
\newcommand\semester{23WS}
\date{20.10.2023}

%select correct course title
%\newcommand\coursetitle{Einführung in die \\ physikalischen Messmethoden}
%\newcommand\coursetitle{Laborübungen 1: \\ Mechanik und Wärme}
\newcommand\coursetitle{Laborübungen 2: \\ Elektrizität, Magnetismus, Optik}
%\newcommand\coursetitle{Fortgeschrittenen Praktikum 1: \\ Technische Physik}
%\newcommand\coursetitle{Fortgeschrittenen Praktikum 2: \\ Allgemeine Physik}

%\input{cover}
%\maketitle %short title alternative

\includepdf[pages={1}]{../Deckblätter/Deckblatt_Gitter.pdf}

\tableofcontents
\newpage

\section{Aufgabenstellung} %jo beschreibn wos gmocht host ------------------------------

Der Versuch Oszillograph geht, wie der Name bereits vermuten lässt, auf die Funktion des Oszilloskopes ein, was in erster Linie die grafische Darstellung elektrischer Spannungen über einen bestimmten Zeitraum beinhaltet.
Mit drei verschiedenen elektrischen Schaltungen soll dies ausprobiert und in diesem Protokoll veranschaulicht werden. 
Die tatächliche Aufgabenstellung sieht hierfür wie folgt aus:

\begin{itemize}
    \item Serienschaltung (Trafo, Kondensator, Widerstand)
    \begin{itemize}
        \item Ermittlung des Phasenversatzes $\phi$
        \item Ermittlung von der Zerfallskonstante $\tau$
    \end{itemize}
    \item Serienschwingkreis (Trafo, Kondensator, Widerstand, Potentiometer)
    \begin{itemize}
        \item Zeichnen der von Kriechfall, Schwingfall, Aperiodischer Grenzfall des Serienschwingkreises
        \item Induktion der Spulte mit und ohne Eisenkern $L_{mitEisenkern}$ / $L_{ohneEisenkern}$
    \end{itemize}
    \item Frequenzbestimmung (Piezo)
    \begin{itemize}
        \item Eigenfrequenz des Stuhles $f_{Stuhl}$
        \item Eigenfrequenz des Piezos $f_{Piezo}$
    \end{itemize}
\end{itemize}

\noindent
Alle Informationen und Methodiken wurden uns von der Technischen Universität bereitgestellt \cite{teachcenter1}. 



\section{Voraussetzungen \& Grundlagen} %Grundlagen erklären, Formeln mit erklärung

    \begin{equation}
        \label{eq:Müll}
        \centerline{$M=\frac{muell}{malle}$}
    \end{equation}

\section{Versuchsanordnung} %mit skizze kurz beschreiben ------------------------------

    \begin{figure}[H]
        \centering
        %\includegraphics[width=0.6\linewidth, angle=-90]{nudes/bild.jpg}
        \caption{müll}
        \label{fig:müllbild}
    \end{figure}

\section{Geräteliste} %jo holt a listn ------------------------------

    \begin{table}[H]
        \centering
        \caption{Im Versuch verwendete Geräte und Utensilien.}
        \label{tab:geraete}
        \begin{tabular}{| l | l | l | l |}
            \hline
            Gerät   & Typ   & Gerätenummer  & Unsicherheit \\
            \hline
        \end{tabular}
    \end{table}


\section{Versuchsdurchführung \& Messergebnisse} %nachvollziehbar und klar dargestellt ------------------------------


\section{Auswertung und Unsicherheitsanalyse} %Nicht nur zahlen angeben ------------------------------

In der Auswertung werden zur erhöhten Genauigkeit durchgehend ungerundete Werte bis zu den Endergebnissen verwendet und nur zur Darstellung gerundet. \\
Zur Berechnung der Unsicherheiten wird, wenn nicht anders angegeben, die Größtunsicherheitsmethode verwendet.


\section{Diskussion} %diskussion der Unsicherheiten und Ergebnisse und evtl. verlgeich mit Literatur ------------------------------


\section{Zusammenfassung} %klare, übersichtliche vollständige beantwortung der Aufgabenstellung ------------------------------


\printbibliography[heading=bibintoc]
\end{document}
